\documentclass[conference]{IEEEtran}
\IEEEoverridecommandlockouts

\title{LAN Collaboration Suite Architecture for Secure Local Networks}

\author{
    \IEEEauthorblockN{Y. Niranjan}
    \IEEEauthorblockA{
        Roll No: CS23B1076\\
        Department of Computer Science and Engineering\\
        Email: cs23b1076@iiitdm.ac.in
    }
    \and
    \IEEEauthorblockN{P. Adithya}
    \IEEEauthorblockA{
        Roll No: CS23B1087\\
        Department of Computer Science and Engineering\\
        Email: cs23b1087@iiitdm.ac.in
    }
}

\begin{document}

\maketitle

\begin{abstract}
This paper presents the architecture of a standalone, server-based LAN collaboration suite designed for secure, offline, multi-user communication. The system supports real-time audio, video, chat, file transfer, and screen sharing over both TCP and UDP protocols, using a hub-and-spoke model optimized for low latency and resilience.
\end{abstract}

\section{Introduction}
The proposed LAN Collaboration Suite is engineered to function entirely within a Local Area Network (LAN) without reliance on external internet connectivity. A centralized server manages client connections, media relaying, data synchronization, and health diagnostics. Each client runs an embedded Python runtime along with a browser-based UI, providing seamless interaction through local WebSockets and HTTP.

\section{System Overview}
The system adopts a client-server architecture where clients connect to a centralized coordination server that handles all services including presence, control messages, and media streaming.

\begin{figure}[h]
\centering
\begin{verbatim}
+-----------+        +-----------+
|  Client A | <--->  |           |
|  (Python) |  UDP   |           |
|  (Web UI) |  TCP   |  Server   |
+-----------+        |           |
                     |           |
+-----------+        |           |
|  Client B | <--->  |           |
|           |        |           |
+-----------+        +-----------+
\end{verbatim}
\caption{Hub-and-Spoke LAN Architecture}
\end{figure}

\section{Communication Channels}
Table \ref{table:channels} summarizes the protocols and transports used.

\begin{table}[h]
\centering
\caption{Communication Channel Overview}
\label{table:channels}
\begin{tabular}{|p{2.2cm}|p{1.2cm}|p{2cm}|p{2cm}|}
\hline
\textbf{Feature} & \textbf{Transport} & \textbf{Direction} & \textbf{Notes} \\
\hline
Audio Conferencing & UDP & Client-Server & Low latency PCM with Opus codec \\
\hline
Video Conferencing & UDP & Client-Server & JPEG/VP8 streams with timestamps \\
\hline
Screen Sharing & TCP & Presenter to Viewers & Reliable PNG frame delivery \\
\hline
Chat & TCP & Bidirectional & JSON structured messages \\
\hline
File Transfer & TCP & Bidirectional & Chunked and resumable \\
\hline
Control Signalling & TCP & Bidirectional & Session presence, reactions, hand raise \\
\hline
Latency Probe & UDP & Bidirectional & Echo-based RTT measurement \\
\hline
UI Bridge & WebSocket & Localhost & Browser to Client Daemon \\
\hline
\end{tabular}
\end{table}

\section{Core Modules}
\subsection{Session Core}
Manages client authentication, presence tracking, heartbeat monitoring, and graceful shutdown control.

\subsection{Chat Service}
Ensures reliable, ordered message synchronization across participants using TCP.

\subsection{Screen Sharing Module}
Provides presenter arbitration and adaptive frame-rate sharing over TCP.

\subsection{Media Mixer}
Handles real-time audio mixing and video frame relaying using per-client UDP sockets.

\section{Technology Stack}
\begin{itemize}
\item Python 3.10+ for cross-platform runtime
\item AsyncIO for high-concurrency networking
\item FastAPI + Uvicorn for embedded web services
\item PyAV, OpenCV, and NumPy for real-time media handling
\item Jinja2 and Vanilla JS for dynamic user interface rendering
\end{itemize}

\section{Security and Resilience}
\begin{itemize}
\item Protected by network firewalls and access control lists
\item Latency telemetry and heartbeat monitoring for health diagnostics
\item Exponential reconnection logic ensures uninterrupted service
\end{itemize}

\section{Conclusion}
The LAN Collaboration Suite delivers an integrated, secure, and resilient communication platform optimized for local networks. Its modular architecture ensures scalability for classrooms, corporate environments, and disaster recovery operations.

\bibliographystyle{IEEEtran}
\bibliography{references}

\end{document}
